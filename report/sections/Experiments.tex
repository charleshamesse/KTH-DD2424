
\section{Experiments}
%In this section, you should
%present the results you achieved with various experiments. The results
%can be presented in tables, plots, etc. 
The project was composed of 8 distinct experiments where each experiment varied from the others either based on what data set was used or what network architecture was used.

\subsection{Inception Score Considerations}

We present a challenge related to the computation and evaluation of the inception score. Most authors evaluate the inception score on 50K GAN-generated images, as recommend by the authors of the original paper \cite{salimans2016improved}. By running a few preliminary experiments, we quickly realized that on top of the actual training of the network, sampling and computing the inception score are also resource-intensive tasks, and sampling 50K images is simply not possible with the time or resources available for this project.


Now, the number of images considered for evaluating the inception score has an impact on this score, as Table \ref{table:exp-isc} depicts. This is due to the fact that the inception score not only evaluates the content of a given image but also the distribution of categories among the whole set of images resulting from the split. In other words, the score is sensitive to the number of images divided by the number of splits. 

\begin{table}[H]
\centering
\setlength{\tabcolsep}{0.5em} % for the horizontal padding

\begin{subtable}{.5\textwidth}
\centering

\begin{tabular}{l l l}
\toprule
Images & Splits & Inception score  \\ 
\midrule
      256  & 5 & 8.13 +- 0.41 \\   
      512  & 5 & 8.04 +- 0.54 \\ 
      1024 & 5 & 9.79 +- 0.36 \\
\bottomrule
\end{tabular}

\end{subtable}% <---- don't forget this %
\begin{subtable}{.5\textwidth}
\centering

\begin{tabular}{l l l}
\toprule
Images & Splits & Inception score  \\ 
\midrule
      256  & 10 & 6.72 +- 0.55 \\   
      512  & 10 & 7.92 +- 0.56\\ 
      1024 & 10 & 8.95 +- 0.44 \\
\bottomrule
\end{tabular}
\end{subtable}%
%
\vspace{0.3cm}
\caption{Inception score for various number of samples of the cifar10 dataset.}
\label{table:exp-isc}
\end{table}%
We choose to stick with 1024 generated images and 5 splits for all of our experiments. With this configuration, we have a target inception score of 9.79. As expected, this is below the claimed inception score of the whole cifar10 dataset, 11.24 \cite{salimans2016improved}. Thus we won't reach state of the art results in terms of inception score, but this isn't an issue since our purpose is to compare various improvements of GAN networks, which isn't affected by this choice. We applied the same calculations to our Reptile data set, resulting in a target inception score of $10.2415905 +- 0.92625165$. Other considerations on the inception score are explained in \cite{barratt2018note}.

%\subsection{Frame of Reference}
%In order to measure the performance of each implementation a frame of reference had to be established. As the golden standard, that every implementation would be compared to, the inception score of the actual data sets were calculated using the number of generated images and splits mentioned above, the resulting scores can be seen in table \ref{table:GS}
%
%\begin{table}[H]
%\centering
%\setlength{\tabcolsep}{0.5em} % for the horizontal padding
%\caption{Inception scores for the data sets}
%\label{table:GS}
%\begin{tabular}{l  l}
%\toprule
%CIFAR-10                   & Reptiles   \\ 
%\midrule
%      9.792953 +- 0.36040735 & 10.2415905 +- 0.92625165 \\   
%
%\bottomrule
%\end{tabular}
%\end{table}



\subsection{DCGAN}
\label{sec:exp-dcgan}

We implement our baseline model in this section. The DCGAN is evaluated on cifar10. We present the evolution of the inception score in Figure \ref{fig:exp-dcgan-is} and both losses in Figure \ref{fig:exp-dcgan-losses}
   
\begin{figure}[t!]
    \centering
    \begin{subfigure}[t]{0.49\textwidth}
        \centering
		\includegraphics[width=\textwidth]{../code/results/figures/dcgan_cifar10_is.png}
		\caption{Inception score}
		\label{fig:exp-dcgan-is}
    \end{subfigure}
    \begin{subfigure}[t]{0.49\textwidth}
        \centering
        \includegraphics[width=\textwidth]{../code/results/figures/dcgan_cifar10_losses.png}
		\caption{Losses}
		\label{fig:exp-dcgan-losses}
    \end{subfigure}
    \caption{DCGAN - training on CIFAR10 over 190 epochs.}
\end{figure}

Losses can hardly be interpreted when treating with GANs, since the generator and discriminator are in a situation of competition where an improvement on the one leads to a deterioration on the other.


\subsection{SN-DCGAN}
\label{sec:exp-sndcgan}
We implement spectral normalization and use it on the discriminator of the DCGAN used in the previous section. We present the evolution of the Inception score and losses in Figures \ref{fig:exp-sndcgan-is} and \ref{fig:exp-sndcgan-losses}, respectively. We observe that both losses are much smoother with spectral normalization. This is the desired result; the original intent of the paper on spectral normalization was to provide a solution to stabilize the training of GANs \cite{miyato2018spectral}. By visual inspection of the generated images we observe a significant reduce in mode collapse and a noticeable improvement of image quality. This did however not result in a significant Inception score improvement.
   
\begin{figure}[H]
    \centering
    \begin{subfigure}[t]{0.49\textwidth}
        \centering
		\includegraphics[width=\textwidth]{../code/results/figures/sndcgan_cifar10_is.png}
		\caption{Inception score}
		\label{fig:exp-sndcgan-is}
    \end{subfigure}
    \begin{subfigure}[t]{0.49\textwidth}
        \centering
        \includegraphics[width=\textwidth]{../code/results/figures/sndcgan_cifar10_losses.png}
		\caption{Losses}
		\label{fig:exp-sndcgan-losses}
    \end{subfigure}
    \caption{SN-DCGAN - training on CIFAR10 over 250 epochs.}
\end{figure}
\subsection{W-DCGAN}
\label{sec:exp-w-dcgan}
We present the evolution of the inception score and losses in Figures \ref{fig:exp-w-dcgan-is} and \ref{fig:exp-w-dcgan-losses}, respectively. We did not expect to get interesting results using this setup since we did not implement any method enforcing the Lipschitz continuity of the discriminator functions, which is required by WGAN. These results coincide with our expectations.
   
\begin{figure}[H]
    \centering
    \begin{subfigure}[t]{0.49\textwidth}
        \centering
		\includegraphics[width=\textwidth]{../code/results/figures/w-dcgan_cifar10_is.png}
		\caption{Inception score}
		\label{fig:exp-w-dcgan-is}
    \end{subfigure}
    \begin{subfigure}[t]{0.49\textwidth}
        \centering
        \includegraphics[width=\textwidth]{../code/results/figures/w-dcgan_cifar10_losses.png}
		\caption{Losses}
		\label{fig:exp-w-dcgan-losses}
    \end{subfigure}
    \caption{W-DCGAN - training on CIFAR10 over 200 epochs.}
\end{figure}

%We observe that both losses are much smoother with spectral normalization. This is the desired result; the original intent of the paper on spectral normalization was to provide a solution to stabilize the training of GANs \cite{miyato2018spectral}. By visual inspection of the generated images we observe a significant reduce in mode collapse and a noticeable improvement of image quality. This did however not result in a significant Inception score improvement.
\subsection{W-WC-DCGAN}
\label{sec:exp-w-wc-dcgan}
We present the evolution of the inception score and losses in Figures \ref{fig:exp-w-wc-dcgan-is} and \ref{fig:exp-w-wc-dcgan-losses}, respectively.
   
\begin{figure}[t!]
    \centering
    \begin{subfigure}[t]{0.49\textwidth}
        \centering
		\includegraphics[width=\textwidth]{../code/results/figures/w-wc-dcgan_cifar10_is.png}
		\caption{Inception score}
		\label{fig:exp-w-wc-dcgan-is}
    \end{subfigure}
    \begin{subfigure}[t]{0.49\textwidth}
        \centering
        \includegraphics[width=\textwidth]{../code/results/figures/w-wc-dcgan_cifar10_losses.png}
		\caption{Losses}
		\label{fig:exp-w-wc-dcgan-losses}
    \end{subfigure}
    \caption{W-WC-DCGAN - training on CIFAR10 over 200 epochs.}
\end{figure}
\subsection{W-SN-DCGAN}
\label{sec:exp-wsndcgan}

We present the evolution of the inception score and losses in Figures \ref{fig:exp-wsndcgan-is} and \ref{fig:exp-wsndcgan-losses}, respectively.
   
\begin{figure}[t!]
    \centering
    \begin{subfigure}[t]{0.49\textwidth}
        \centering
		\includegraphics[width=\textwidth]{../code/results/figures/w-sn-dcgan_cifar10_is.png}
		\caption{Inception score}
		\label{fig:exp-w-sn-dcgan-is}
    \end{subfigure}
    \begin{subfigure}[t]{0.49\textwidth}
        \centering
        \includegraphics[width=\textwidth]{../code/results/figures/w-sn-dcgan_cifar10_losses.png}
		\caption{Losses}
		\label{fig:exp-w-sn-dcgan-losses}
    \end{subfigure}
    \caption{W-SN-DCGAN - training on CIFAR10 over 200 epochs.}
\end{figure}



\todo{to justify the use of WGAN / the need for loss smoothness: Plotting these learning curves is not only useful for debugging and hyperparameter searches, but also correlate remarkably well with the observed sample quality. (taken from WGAN paper) }
\subsection{Performance Comparison}

\begin{figure}[h]
\centering
\includegraphics[width=\textwidth]{../code/results/figures/all_cifar10_is.png}
\caption{All evaluated models - Inception score, training on cifar10}
\label{fig:exp-all-is}
\end{figure}

