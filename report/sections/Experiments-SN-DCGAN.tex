\subsection{SN-DCGAN}
\label{sec:exp-sndcgan}
We implement spectral normalization and use it on the discriminator of the DCGAN used in the previous section. We present the evolution of the Inception score and losses in Figures \ref{fig:exp-sndcgan-is} and \ref{fig:exp-sndcgan-losses}, respectively. We observe that both losses are much smoother with spectral normalization. This is the desired result; the original intent of the paper on spectral normalization was to provide a solution to stabilize the training of GANs \cite{miyato2018spectral}. By visual inspection of the generated images we observe a significant reduce in mode collapse and a noticeable improvement of image quality. This did however not result in a significant Inception score improvement.
   
\begin{figure}[H]
    \centering
    \begin{subfigure}[t]{0.49\textwidth}
        \centering
		\includegraphics[width=\textwidth]{../code/results/figures/sndcgan_cifar10_is.png}
		\caption{Inception score}
		\label{fig:exp-sndcgan-is}
    \end{subfigure}
    \begin{subfigure}[t]{0.49\textwidth}
        \centering
        \includegraphics[width=\textwidth]{../code/results/figures/sndcgan_cifar10_losses.png}
		\caption{Losses}
		\label{fig:exp-sndcgan-losses}
    \end{subfigure}
    \caption{SN-DCGAN - training on CIFAR10 over 250 epochs.}
\end{figure}