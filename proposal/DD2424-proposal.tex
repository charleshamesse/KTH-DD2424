
\documentclass[a4paper]{article}

%% Language and font encodings
\usepackage[english]{babel}
\usepackage[T1]{fontenc}

%% Sets page size and margins
\usepackage[a4paper,top=3cm,bottom=2cm,left=3cm,right=3cm,marginparwidth=1.75cm]{geometry}

%% Useful packages
\usepackage{amsmath}
\usepackage{graphicx}
\usepackage[colorinlistoftodos]{todonotes}
\usepackage[colorlinks=true, allcolors=blue]{hyperref}

\title{Animal Morphing using Deep Convolutional Generative Adversarial Networks}
\author{
	You
}

\begin{document}
\maketitle

%\begin{abstract}
%Your abstract.
%\end{abstract}

\section{Introduction}
In this project, we will implement Deep Convolutional Generative Adversarial Networks (DCGANs) for image processing. We will use a subset of CIFAR-100 dataset containing only animals. Our aim will be three-fold; we will first implement a photo-realistic animal generator, then allow users to sketch animals and have this sketch morphed into a photo-realistic animal, then finally allow users to restrict the class of the generated animal.


\section{Random Generator}

\section{Sketch-to-Picture Generator}


\section{Constrained Sketch-to-Picture Generator}


\end{document}